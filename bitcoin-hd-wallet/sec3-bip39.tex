\section{Mnemonic code for generating deterministic keys(BIP39)}
BIP39提出了一种从助记词派生HD钱包seed的方法,使得用户只需要掌握这组助记词即可派生出HD钱包的密钥树,
从而优化用户的交互体验。  
该协议主要包含两个过程:1.从entropy生成助记词的过程;2.从助记词派生seed的过程。
其中,助记词是从事先定义好的wordlist里选出的符合相应长度要求的单词集合,
目前针对英语、日语、韩语、西班牙语、中文(简体/繁体)、法语、意大利语都有相应的wordlist,
在使用前都对它们进行UTF-8编码处理,详见Wordlists\footnote{\url{https://github.com/bitcoin/bips/blob/master/bip-0039/bip-0039-wordlists.md}}。


\subsection{Entropy $\rightarrow$ Mnemonic}
用来产生助记词的entropy长度范围为128-512bit,但必须是32的整数倍,entropy长度越长,
安全性就越高,同时生成助记词的个数也越多。助记词的产生过程:
假设初始的熵值e的长度为ENT,取$SHA256(e)$的前$ENT/32$bit作为检验和附在e的后面,
随后将级联后的比特串按11bit为基本单元进行分组,即每个分组的数字范围是0-2047,
将其作为index从wordlist中读取相应的word,最终选出的词语即构成了一个mnemonic sentence。

下面的列表展示了初始熵值长度ENT,校验和长度与生成的mnemonic sentence长度之间的关系:
		$$CS = ENT / 32$$
		$$MS = (ENT + CS) / 11$$
\begin{table}
\centering
\begin{tabular}{|c|c|c|c|c|}
\hline
\small
ENT &  CS  &   ENT+CS  &  MS  \\\hline
128 &  4  &  132 &  12  \\\hline
160 &  5  &  165 &  15 \\\hline
192 &  6  &  198 &  18 \\\hline
224  &  7  &  231 &  21 \\\hline
256  &  8  &  264 &  24 \\\hline
\end{tabular}
\end{table}

\subsection{Mnemonic $\rightarrow$ Seed}
BIP39规定使用PBKDF2函数来生成seed,PBKDF2各参数设置如下:mnemonic sentence以 UTF-8 NFKD编码后作为password,“mnemonic”联接passphrase同样使用UTF-8 NFKD
进行编码后作为$salt$,$c=2048, PRF=HAMC-SHA512, dklen=512$。
最终该算法输出的512bit作为 seed,使用BIP32派生HD wallet。


以下测试代码基于Trezor的python-mnemonic库\footnote{\url{https://github.com/trezor/python-mnemonic}}:

\begin{lstlisting}[language=python]
def main():
    import binascii
    import sys
    from bip32utils import BIP32Key
    if len(sys.argv) > 1:
        data = sys.argv[1]
    else:
        data = sys.stdin.readline().strip()
    data = binascii.unhexlify(data)
    m = Mnemonic('english')
    code=m.to_mnemonic(data)
    seed=m.to_seed(code,'TREZOR')
    xprv = BIP32Key.fromEntropy(seed).ExtendedKey()
    print('mnemonic : %s (%d words)' % (code, len(code.split(' '))))
    print('seed     : %s (%d bits)' % (binascii.hexlify(seed),len(seed) * 4))
    print('xprv     : %s' % xprv)
\end{lstlisting}

